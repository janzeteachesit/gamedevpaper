\section{Conclusion}
Video game development presents several unique challenges. The requirements documents have to take the intent of the designer into account, and many of the requirements are very difficult to quantify. Emotional requirements are one unique challenge, and effectively quantifying them is a current research area. 

The game industry has already moved from hierarchical tree based inheritance to composition for larger projects. This lead to the entity-component design paradigm which focuses on entities, components, and system-managers. High performance constraints and the move to multicore systems is currently pushing the game industry towards functional programming as well. By adopting functional programming, video games will be able to take full advantage of multicore systems.

Developers want to make sure that the users of their games become immersed in them, and very small bugs can break that immersion. This means that video games must have very few bugs. In order to reduce the number of bugs, video games will typically have functionality , compliance, compatibility, multiplayer, localization, soak, regression and beta testing.

Video games also have unique user control devices including the keyboard, mouse, joysticks, gamepads, touchscreens, etc. Feedback mechanisms in video games are also very important, and ``lag'' in video games can completely destroy the user experience. 

Finally, there are several different frameworks that developers can choose from when developing video games. Some larger companies make their own, whereas smaller companies usually choose from those that have already been built. Each of the frameworks have pros and cons and developers must choose those frameworks that complement the work they are doing the most. All of these challenges make video game development very difficult and distinct from typical software development.
